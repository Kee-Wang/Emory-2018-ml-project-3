\documentclass[11pt]{article}
\usepackage{graphicx}
\marginparwidth 0pt
\oddsidemargin  0pt
\evensidemargin  0pt
\marginparsep 0pt
\usepackage{amsmath}
\topmargin 0pt
\usepackage{mathtools}
\textwidth   6.5 in
 \textheight  8 in
\usepackage{fancyhdr}
\renewcommand{\headrulewidth}{2pt}
\renewcommand{\footrulewidth}{1pt}
 
\pagestyle{fancy}
\usepackage{amsmath}
 %%%%%%% %%%%%%% %%%%%%% %%%%%%% %%%%%%% %%%%%%%For python%%%
 % Default fixed font does not support bold face
\DeclareFixedFont{\ttb}{T1}{txtt}{bx}{n}{12} % for bold
\DeclareFixedFont{\ttm}{T1}{txtt}{m}{n}{12}  % for normal

% Custom colors
\usepackage{color}
\definecolor{deepblue}{rgb}{0,0,0.5}
\definecolor{deepred}{rgb}{0.6,0,0}
\definecolor{deepgreen}{rgb}{0,0.5,0}

\usepackage{listings}

% Python style for highlighting
\newcommand\pythonstyle{\lstset{
language=Python,
breaklines=true,
basicstyle=\ttm,
otherkeywords={self},             % Add keywords here
keywordstyle=\ttb\color{deepblue},
emph={MyClass,__init__},          % Custom highlighting
emphstyle=\ttb\color{deepred},    % Custom highlighting style
stringstyle=\color{deepgreen},
frame=tb,                         % Any extra options here
showstringspaces=false            % 
}}

%++++++Hyperlink
\usepackage{hyperref}
\hypersetup{
    colorlinks=true,
    linkcolor=blue,
    filecolor=magenta,      
    urlcolor=cyan,
}


% Python environment
\lstnewenvironment{python}[1][]
{
\pythonstyle
\lstset{#1}
}
{}

% Python for external files
\newcommand\pythonexternal[2][]{{
\pythonstyle
\lstinputlisting[#1]{#2}}}

% Python for inline
\newcommand\pythoninline[1]{{\pythonstyle\lstinline!#1!}}
%%%%%%%%%%%%%%%%%%%%%%%%%%%%%%%% %%%%%%% %%%%%%% %%%%%%%


%\pagestyle{empty}

\begin{document}


\begin{center}
  {\bf
  CS 534 Machine Learning  }

\vspace{12pt}

  Project 3, due Monday, April 16
  \\
  Chenxi Cai \\
  Yifei Ren \\
  Qingfeng (Kee) Wang
 
\end{center}

\vspace{12pt}




\clearpage
(Notice, all links (table of contents, figures references, table references etc.) are clickable! You can download this PDF and open it using Preview or Adobe Reader to enable this convenient feature.)
 
\tableofcontents{}

\clearpage
\section{This is first section}
I said nothing.
\subsection{This is a subsection section.}




\begin{table}[!htb]
\centering
\caption{This is a table.}
\label{tab:q2}
\begin{tabular}{lllll}
\hline \hline

Iteration	&3	&5	&10		&20 \\ \hline

Error rate(\%)  &16.67	&10.0	&10.0		&13.3	\\ \hline \hline

\end{tabular}
\end{table}





\clearpage
\section{Source codes}

Finally, codes are enclosed here.

\pythonexternal{./../project3.py}




\vspace{24pt}



\end{document}

















